\documentclass{beamer}

\mode<presentation> {
  %\usetheme{default}
  %\usetheme{AnnArbor}
  %\usetheme{Antibes}
  %\usetheme{Bergen}
  %\usetheme{Berkeley}
  %\usetheme{Berlin}
  \usetheme{Boadilla}
  %\usetheme{CambridgeUS}
  %\usetheme{Copenhagen}
  %\usetheme{Darmstadt}
  %\usetheme{Dresden}
  %\usetheme{Frankfurt}
  %\usetheme{Goettingen}
  %\usetheme{Hannover}
  %\usetheme{Ilmenau}
  %\usetheme{JuanLesPins}
  %\usetheme{Luebeck}
  %\usetheme{Madrid}
  %\usetheme{Malmoe}
  %\usetheme{Marburg}
  %\usetheme{Montpellier}
  %\usetheme{PaloAlto}
  %\usetheme{Pittsburgh}
  %\usetheme{Rochester}
  %\usetheme{Singapore}
  %\usetheme{Szeged}
  %\usetheme{Warsaw}

  %\usecolortheme{albatross}
  %\usecolortheme{beaver}
  %\usecolortheme{beetle}
  %\usecolortheme{crane}
  %\usecolortheme{dolphin}
  %\usecolortheme{dove}
  %\usecolortheme{fly}
  %\usecolortheme{lily}
  %\usecolortheme{orchid}
  %\usecolortheme{rose}
  %\usecolortheme{seagull}
  %\usecolortheme{seahorse}
  \usecolortheme{whale}
  %\usecolortheme{wolverine}

  %\setbeamertemplate{footline} % To remove the footer line in all slides uncomment this line
  %\setbeamertemplate{footline}[page number]
  % To replace the footer line in all slides with a simple slide count uncomment this line
  \setbeamertemplate{navigation symbols}{}
  % To remove the navigation symbols from the bottom of all slides uncomment this line
}

\usepackage{graphicx} % Allows including images
\usepackage{booktabs} % Allows the use of \toprule, \midrule and \bottomrule in tables
\usepackage{times}
\usepackage[T1]{fontenc}

%%%%%%%%%%%%%%%%%%%%%%%%%%%%%%%%%%%%%%%%%%%%%%%%%%%%%%%%%%%%%%%%%%%%%%%%%%%%%%%%%%%%%%%%%%%%%%%%%%%
\newcommand{\python}         {\textit{Python}}
\newcommand{\bash}           {\textit{Bash}}
\newcommand{\debian}         {\textit{Debian GNU/Linux}}
\newcommand{\fpgaprepare}    {\textit{fpga\_prepare}}
\newcommand{\fpgamakesyn}    {\textit{fpga\_make\_syn}}
\newcommand{\fpgaprog}       {\textit{fpga\_prog}}
\newcommand{\console}        {\textit{console}}

\newcommand{\shellcmd}[1]{\\\indent\indent\texttt{\scriptsize #1}\\}
%%%%%%%%%%%%%%%%%%%%%%%%%%%%%%%%%%%%%%%%%%%%%%%%%%%%%%%%%%%%%%%%%%%%%%%%%%%%%%%%%%%%%%%%%%%%%%%%%%%

\title[FPGA Helpers]{Tutorial: FPGA Helpers v0.1.1}

\author{
  Rodrigo A. Melo\\
  \textit{rodrigomelo9@gmail.com}\\
  \textit{ar.linkedin.com/in/rodrigoalejandromelo}
}

%%%%%%%%%%%%%%%%%%%%%%%%%%%%%%%%%%%%%%%%%%%%%%%%%%%%%%%%%%%%%%%%%%%%%%%%%%%%%%%%%%%%%%%%%%%%%%%%%%%

\begin{document}

\begin{frame}
  \titlepage
\end{frame}

\section{Intro}

\begin{frame}{FPGA Helpers}
  \begin{exampleblock}{What is this?}
    A bunch of scripts to use FPGA development tools in a vendor independent way.
  \end{exampleblock}
  \begin{block}{Development}
    \python\ and \bash\ scripts over \debian.
  \end{block}
\end{frame}

\begin{frame}{Current components}
  \begin{itemize}
    \item \textbf{\fpgaprepare:} used to set the system to execute the tools of the vendors.
    \item \textbf{\fpgamakesyn:} generates a Makefile and two TCL files, which are used to
          configure and execute a synthesis project.
    \item \textbf{\fpgaprog:} used to transfer a bitstream to a FPGA or memory.
  \end{itemize}
\end{frame}

\begin{frame}{Prerequisites}
  FPGA Helpers assume that the vendor's tool is correctly installed, is well configured (license)
  and is available in the system path. For the last point yo need to:
  \begin{itemize}
    \item Manually add the path of vendor's binaries in the system path and run the license
          manager when necessary; Or ...
    \item Automate this task in for example .bashrc; Or ...
    \item Run \fpgaprepare.
  \end{itemize}
\end{frame}

\begin{frame}{FPGA Prepare}
  \begin{columns}
    \column{0.7\textwidth}
      \begin{exampleblock}{}
        Prepare a \console\ to run vendor tools as installed in the system,
        which is a requirement to run \fpgaprog\ or the Makefile generated by \fpgamakesyn.
      \end{exampleblock}
      \fpgaprepare\ allow to configure paths and some other needed parameters (which are
      stored on ~/.fpga\_helpers):
      \shellcmd{\$ fpga\_prepare -{}-config-all}
      To prepare a \console\ to run all the supported vendors tools:
      \shellcmd{\$ fpga\_prepare -{}-all}
      Or to choose interactively:
      \shellcmd{\$ fpga\_prepare}
    \column{0.3\textwidth}
      Supported vendors tools:
      \begin{itemize}
        \item Xilinx ISE and Vivado
        \item Altera Quartus2
        \item Microsemi Libero-SoC
      \end{itemize}
      To see individual config and run options:
      \shellcmd{\$ fpga\_prepare -h}
  \end{columns}
\end{frame}

%%%%%%%%%%%%%%%%%%%%%%%%%%%%%%%%%%%%%%%%%%%%%%%%%%%%%%%%%%%%%%%%%%%%%%%%%%%%%%%%%%%%%%%%%%%%%%%%%%%
\end{document}
