\documentclass{beamer}

\mode<presentation> {
  %\usetheme{default}
  %\usetheme{AnnArbor}
  %\usetheme{Antibes}
  %\usetheme{Bergen}
  %\usetheme{Berkeley}
  %\usetheme{Berlin}
  \usetheme{Boadilla}
  %\usetheme{CambridgeUS}
  %\usetheme{Copenhagen}
  %\usetheme{Darmstadt}
  %\usetheme{Dresden}
  %\usetheme{Frankfurt}
  %\usetheme{Goettingen}
  %\usetheme{Hannover}
  %\usetheme{Ilmenau}
  %\usetheme{JuanLesPins}
  %\usetheme{Luebeck}
  %\usetheme{Madrid}
  %\usetheme{Malmoe}
  %\usetheme{Marburg}
  %\usetheme{Montpellier}
  %\usetheme{PaloAlto}
  %\usetheme{Pittsburgh}
  %\usetheme{Rochester}
  %\usetheme{Singapore}
  %\usetheme{Szeged}
  %\usetheme{Warsaw}

  %\usecolortheme{albatross}
  %\usecolortheme{beaver}
  %\usecolortheme{beetle}
  %\usecolortheme{crane}
  %\usecolortheme{dolphin}
  %\usecolortheme{dove}
  %\usecolortheme{fly}
  %\usecolortheme{lily}
  %\usecolortheme{orchid}
  %\usecolortheme{rose}
  %\usecolortheme{seagull}
  %\usecolortheme{seahorse}
  \usecolortheme{whale}
  %\usecolortheme{wolverine}

  %\setbeamertemplate{footline} % To remove the footer line in all slides uncomment this line
  %\setbeamertemplate{footline}[page number]
  % To replace the footer line in all slides with a simple slide count uncomment this line
  \setbeamertemplate{navigation symbols}{}
  % To remove the navigation symbols from the bottom of all slides uncomment this line
}

\usepackage{graphicx} % Allows including images
\usepackage{booktabs} % Allows the use of \toprule, \midrule and \bottomrule in tables
\usepackage{times}
\usepackage[T1]{fontenc}
\usepackage{verbatim}

%%%%%%%%%%%%%%%%%%%%%%%%%%%%%%%%%%%%%%%%%%%%%%%%%%%%%%%%%%%%%%%%%%%%%%%%%%%%%%%%%%%%%%%%%%%%%%%%%%%
\newcommand{\python}         {\textit{Python}}
\newcommand{\bash}           {\textit{Bash}}
\newcommand{\debian}         {\textit{Debian GNU/Linux}}
\newcommand{\fpgasetup}      {\textit{fpga\_setup}}
\newcommand{\fpgasynt}       {\textit{fpga\_synt}}
\newcommand{\fpgaprog}       {\textit{fpga\_prog}}
\newcommand{\console}        {\textit{console}}
\newcommand{\makefile}       {\textit{Makefile}}

\newcommand{\shellcmd}[1]{\\\indent\indent\texttt{\scriptsize #1}\\}
%%%%%%%%%%%%%%%%%%%%%%%%%%%%%%%%%%%%%%%%%%%%%%%%%%%%%%%%%%%%%%%%%%%%%%%%%%%%%%%%%%%%%%%%%%%%%%%%%%%

\title[FPGA Helpers]{Tutorial: FPGA Helpers v0.1.1}

\author{
  Rodrigo A. Melo\\
  \textit{rodrigomelo9@gmail.com}\\
  \textit{ar.linkedin.com/in/rodrigoalejandromelo}
}

%%%%%%%%%%%%%%%%%%%%%%%%%%%%%%%%%%%%%%%%%%%%%%%%%%%%%%%%%%%%%%%%%%%%%%%%%%%%%%%%%%%%%%%%%%%%%%%%%%%

\begin{document}

\begin{frame}
  \titlepage
\end{frame}

\section{Intro}

\begin{frame}{FPGA Helpers}
  \begin{exampleblock}{What is this?}
    A bunch of scripts to use FPGA development tools in a vendor independent way.
  \end{exampleblock}
  \begin{block}{Development}
    \python\ and \bash\ scripts over \debian.
  \end{block}
\end{frame}

\begin{frame}{Current components}
  \begin{itemize}
    \item \textbf{\fpgasetup:} used to set the system to execute the tools of the vendors.
    \item \textbf{\fpgasynt:} generates a Makefile and two TCL files, which are used to
      configure and execute a synthesis project.
    \item \textbf{\fpgaprog:} used to transfer a bitstream to a FPGA or memory.
  \end{itemize}
\end{frame}

%% FPGA Setup %%%%%%%%%%%%%%%%%%%%%%%%%%%%%%%%%%%%%%%%%%%%%%%%%%%%%%%%%%%%%%%%%%%%%%%%%%%%%%%%%%%%%

\section{FPGA Setup}

\begin{frame}[fragile]{FPGA Setup (I)}
  \begin{columns}
    \column{0.75\textwidth}
      \begin{block}{}
        \small
        \fpgaprog\ and the \makefile\ generated by \fpgasynt\ uses vendors tools.
        They assumes that are well installed and configured (license) and the
        binaries must be in the system path. It could be done manually,
        automated in for example \textit{.bashrc} or using \fpgasetup.
      \end{block}{}
    \column{0.25\textwidth}
      \small Supported tools:
      \begin{itemize}
        \tiny
        \item Xilinx ISE
        \item Xilinx Vivado
        \item Altera Quartus2
        \item Microsemi Libero-SoC
      \end{itemize}
  \end{columns}
  \scriptsize
  \verbatiminput{temp/fpgasetup1.txt}
\end{frame}

\begin{frame}[fragile]{FPGA Setup (II)}
  It can run interactively:
  \scriptsize
  \verbatiminput{temp/fpgasetup2.txt}
  \normalsize
  About configurations:
  \begin{itemize}
    \item Are related to paths and licenses servers (when needed)
    \item Has default values and are filled interactively
    \item The \textit{tab key} can be used to autocomplete paths and files
    \item Data is saved on <HOME>/.fpga\_helpers file
  \end{itemize}
\end{frame}

%% FPGA Synt %%%%%%%%%%%%%%%%%%%%%%%%%%%%%%%%%%%%%%%%%%%%%%%%%%%%%%%%%%%%%%%%%%%%%%%%%%%%%%%%%%%%%%

\section{FPGA Synt}

\begin{frame}[fragile]{FPGA Synt (I)}
  \begin{columns}
    \column{0.75\textwidth}
      \begin{block}{}
        \small
        It generates a \makefile\ a TCL file for options and one or more TCL files
        with vendor specific commands to run the synthesis process.
      \end{block}{}
    \column{0.25\textwidth}
      \small Supported tools:
      \begin{itemize}
        \tiny
        \item Xilinx ISE
        \item Xilinx Vivado
        \item Altera Quartus2
      \end{itemize}
  \end{columns}
  \scriptsize
  \verbatiminput{temp/fpgasynt.txt}
\end{frame}

\begin{frame}[fragile]{FPGA Synt (II) - Examples}
  \small
  \verbatiminput{temp/fpgasynt_ex1.txt}
  \verbatiminput{temp/fpgasynt_ex2.txt}
\end{frame}

\begin{frame}[fragile]{FPGA Synt (III) - Makefile}
  \tiny
  \verbatiminput{temp/makefile.txt}
  \begin{alertblock}{}
    \small
    The \makefile\ uses vendors tools, so you could run \fpgasetup\ when open
    a new \console.
  \end{alertblock}
\end{frame}

\begin{frame}[fragile]{FPGA Synt (IV) - options.tcl}
  \tiny
  \verbatiminput{temp/options.txt}
\end{frame}

\begin{frame}[fragile]{FPGA Synt (V) - vendors TCL files}
  %\tiny
  \begin{itemize}
    \item The commands \textit{fpga\_device} and \textit{fpga\_file}, as well
          as the value \textit{\$FPGA\_TOOL} are implemented in each vendor TCL
          file.
    \item If a project file exists, it is used and options.tcl is omitted.
          The name of the project file must be different of the tool name
          (it is used by the vendors TCL files).
    \item Vendors TCL files generates log files called
          TOOLNAME-OPERATION-OPTIMIZATION.log, where OPERATION could be syn or
          imp and OPTIMIZATION could be user, area, power and speed.
          Examples are \textit{ise-syn-area.log} and \textit{quartus2-imp-speed.log}.
  \end{itemize}
\end{frame}

%% FPGA Prog %%%%%%%%%%%%%%%%%%%%%%%%%%%%%%%%%%%%%%%%%%%%%%%%%%%%%%%%%%%%%%%%%%%%%%%%%%%%%%%%%%%%%%

\section{FPGA Prog}

\begin{frame}[fragile]{FPGA Prog (I)}
  \begin{columns}
    \column{0.75\textwidth}
      \begin{block}{}
        \small
        Program to transfer a bitstream to the desired (supported) device.
      \end{block}{}
    \column{0.25\textwidth}
      \small Supported tools:
      \begin{itemize}
        \tiny
        \item Xilinx ISE (\textcolor{teal}{FPGA}, \textcolor{teal}{SPI}, \textcolor{teal}{BPI}, \textcolor{teal}{One XCF})
        \item Altera Quartus2 (\textcolor{teal}{FPGA})
      \end{itemize}
  \end{columns}
  \tiny
  \begin{verbatim}
$ fpga_prog --help
usage: fpga_prog [-h] [-v] [-t TOOLNAME] [-d DEVICE] [-p POSITION]
                    [-m MEMNAME] [-w WIDTH] [-b BOARDNAME|BOARDFILE]
                    BITSTREAM

Transfers a BitStream to a device.

positional arguments:
  BITSTREAM             BitStream to be transferred

optional arguments:
  -h, --help            show this help message and exit
  -v, --version         show program's version number and exit
  -t TOOLNAME, --tool TOOLNAME
                        Name of the vendor tool to be used [ise |quartus2]
  \end{verbatim}
\end{frame}

\begin{frame}[fragile]{FPGA Prog (II)}
  \tiny
  \begin{verbatim}
device arguments:
  -d DEVICE, --device DEVICE
                        Type of the target device
                        [fpga(default)|spi|bpi|xcf|detect|unlock]
  -p POSITION, --position POSITION
                        positive number which represents the POSITION of the
                        device in the JTAG chain [1]
  -m MEMNAME, --memname MEMNAME
                        Name of the target memory (when applicable)
                        [UNDEFINED]
  -w WIDTH, --width WIDTH
                        positive number which representes the WIDTH of bits of
                        the target memory (when applicable) [1]
  -b BOARDNAME|BOARDFILE, --board BOARDNAME|BOARDFILE
                        Name of a supported board or file (.yaml) of a
                        new/custom board (note: if you use the board option,
                        -p, -m, -w and -t will be overwritten) []

Supported boards: avnet_s6micro, digilent_atlys, gaisler_xc6s,
terasic_de0nano, xilinx_sp601
  \end{verbatim}
  \begin{alertblock}{}
    \small
    \fpgaprog\ uses vendors tools, so you could run \fpgasetup\ when open
    a new \console.
  \end{alertblock}
\end{frame}

%%%%%%%%%%%%%%%%%%%%%%%%%%%%%%%%%%%%%%%%%%%%%%%%%%%%%%%%%%%%%%%%%%%%%%%%%%%%%%%%%%%%%%%%%%%%%%%%%%%

\section{License}

\begin{frame}{Tutorial License}
  \begin{figure}[!t]
    \includegraphics[width=0.2\textwidth]{../images/cc-by-sa.png}
  \end{figure}
  \centering
  This work is licensed under a Creative Commons Attribution-ShareAlike 4.0 International License.
  \url{https://creativecommons.org/licenses/by-sa/4.0/}
\end{frame}

\end{document}
