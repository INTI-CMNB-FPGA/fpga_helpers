\documentclass{beamer}

\mode<presentation> {
  %\usetheme{default}
  %\usetheme{AnnArbor}
  %\usetheme{Antibes}
  %\usetheme{Bergen}
  %\usetheme{Berkeley}
  %\usetheme{Berlin}
  \usetheme{Boadilla}
  %\usetheme{CambridgeUS}
  %\usetheme{Copenhagen}
  %\usetheme{Darmstadt}
  %\usetheme{Dresden}
  %\usetheme{Frankfurt}
  %\usetheme{Goettingen}
  %\usetheme{Hannover}
  %\usetheme{Ilmenau}
  %\usetheme{JuanLesPins}
  %\usetheme{Luebeck}
  %\usetheme{Madrid}
  %\usetheme{Malmoe}
  %\usetheme{Marburg}
  %\usetheme{Montpellier}
  %\usetheme{PaloAlto}
  %\usetheme{Pittsburgh}
  %\usetheme{Rochester}
  %\usetheme{Singapore}
  %\usetheme{Szeged}
  %\usetheme{Warsaw}

  %\usecolortheme{albatross}
  %\usecolortheme{beaver}
  %\usecolortheme{beetle}
  %\usecolortheme{crane}
  %\usecolortheme{dolphin}
  %\usecolortheme{dove}
  %\usecolortheme{fly}
  %\usecolortheme{lily}
  %\usecolortheme{orchid}
  %\usecolortheme{rose}
  %\usecolortheme{seagull}
  %\usecolortheme{seahorse}
  \usecolortheme{whale}
  %\usecolortheme{wolverine}

  %\setbeamertemplate{footline} % To remove the footer line in all slides uncomment this line
  %\setbeamertemplate{footline}[page number]
  % To replace the footer line in all slides with a simple slide count uncomment this line
  \setbeamertemplate{navigation symbols}{}
  % To remove the navigation symbols from the bottom of all slides uncomment this line
}

\usepackage{graphicx} % Allows including images
\usepackage{booktabs} % Allows the use of \toprule, \midrule and \bottomrule in tables
\usepackage{times}
\usepackage[T1]{fontenc}
\usepackage{verbatim}

%%%%%%%%%%%%%%%%%%%%%%%%%%%%%%%%%%%%%%%%%%%%%%%%%%%%%%%%%%%%%%%%%%%%%%%%%%%%%%%%%%%%%%%%%%%%%%%%%%%
\newcommand{\python}         {\textit{Python}}
\newcommand{\bash}           {\textit{Bash}}
\newcommand{\debian}         {\textit{Debian GNU/Linux}}
\newcommand{\fpgasetup}      {\textit{fpga\_setup}}
\newcommand{\fpgawizard}     {\textit{fpga\_wizard}}
\newcommand{\fpgasynt}       {\textit{fpga\_synt}}
\newcommand{\fpgaprog}       {\textit{fpga\_prog}}
\newcommand{\fpgadeps}       {\textit{fpga\_deps}}
\newcommand{\console}        {\textit{console}}
\newcommand{\makefile}       {\textit{Makefile}}

\newcommand{\shellcmd}[1]{\\\indent\indent\texttt{\scriptsize #1}\\}

\newcommand{\version}{\input{temp/version}}

%%%%%%%%%%%%%%%%%%%%%%%%%%%%%%%%%%%%%%%%%%%%%%%%%%%%%%%%%%%%%%%%%%%%%%%%%%%%%%%%%%%%%%%%%%%%%%%%%%%

\title[FPGA Helpers]{Tutorial: FPGA Helpers v\version}

\author{
  Rodrigo A. Melo\\
  \textit{rodrigomelo9@gmail.com}\\
  \textit{ar.linkedin.com/in/rodrigoalejandromelo}
}

%%%%%%%%%%%%%%%%%%%%%%%%%%%%%%%%%%%%%%%%%%%%%%%%%%%%%%%%%%%%%%%%%%%%%%%%%%%%%%%%%%%%%%%%%%%%%%%%%%%

\begin{document}

\begin{frame}
  \titlepage
\end{frame}

\section{Intro}

\begin{frame}{FPGA Helpers}
  \begin{exampleblock}{What is this?}
    A Free Software project which consist on a bunch of scripts which helps to use FPGA development
    tools in a vendor independent way.
  \end{exampleblock}
  \begin{block}{Development}
    \python\ and \bash\ scripts over \debian.
  \end{block}
\end{frame}

\begin{frame}{Current components}
  \begin{itemize}
    \item \textbf{\fpgasetup:} set the system to execute the vendor's tool [Linux].
    \item \textbf{\fpgawizard:} generates options.tcl and Makefile (project files).
    \item \textbf{\fpgasynt:} run synthesis based on the project file of the vendor's tool.
    \item \textbf{\fpgaprog:} transfer a bitstream to a FPGA or memory.
    \item \textbf{\fpgadeps:} collect HDL files of the project [WIP].
  \end{itemize}
\end{frame}

%% FPGA Setup %%%%%%%%%%%%%%%%%%%%%%%%%%%%%%%%%%%%%%%%%%%%%%%%%%%%%%%%%%%%%%%%%%%%%%%%%%%%%%%%%%%%%

\section{FPGA Setup}

\begin{frame}[fragile]{FPGA Setup (I)}
  \begin{block}{}
    The \makefile\ which run the Tcl scripts for synthesis and programming assumes that the
    vendor's tool is well configured and available in the system path.
    In a GNU/Linux system, it could be done manually, automated in for example \textit{.bashrc} or
    using \fpgasetup.
  \end{block}{}
  \scriptsize
  \verbatiminput{temp/setup_help}
\end{frame}

\begin{frame}[fragile]{FPGA Setup (II)}
  It also can be run interactively:
  \scriptsize
  \verbatiminput{temp/setup_run}
  \normalsize
  About configurations:
  \begin{itemize}
    \item Are related to paths and licenses servers (when needed)
    \item Has default values and are filled interactively
    \item The \textit{tab key} can be used to autocomplete paths and files
    \item Data is saved on <HOME>/.fpga\_helpers file
  \end{itemize}
\end{frame}

%% FPGA Wizard %%%%%%%%%%%%%%%%%%%%%%%%%%%%%%%%%%%%%%%%%%%%%%%%%%%%%%%%%%%%%%%%%%%%%%%%%%%%%%%%%%%%

\section{FPGA Wizard}

\begin{frame}[fragile]{FPGA Wizard (I)}
  \begin{block}{}
    Generates options.tcl and Makefile (project files).
  \end{block}{}
  \tiny
  \verbatiminput{temp/wizard}
\end{frame}

\begin{frame}[fragile]{FPGA Wizard (II)}
  \scriptsize
  options.tcl
  \begin{verbatim}
set fpga_name XC6SLX9-2-CSG324
set fpga_pos  1

fpga_device   $fpga_name

fpga_file     core_file.vhdl    -lib LIB_NAME
fpga_file     package_file.vhdl -lib LIB_NAME
fpga_file     top_file.vhdl     -top TOP_NAME
  \end{verbatim}
  Makefile
  \begin{verbatim}
#!/usr/bin/make

TOOL    = vivado
TCLPATH = project
include $(TCLPATH)/Makefile
  \end{verbatim}
\end{frame}

%% FPGA Synt %%%%%%%%%%%%%%%%%%%%%%%%%%%%%%%%%%%%%%%%%%%%%%%%%%%%%%%%%%%%%%%%%%%%%%%%%%%%%%%%%%%%%%

\section{FPGA Synt}

\begin{frame}[fragile]{FPGA Synt (I)}
  \begin{block}{}
    You can make a project using the vendor's tool GUI, and after that, use FPGA Synt to run
    synthesis, implementation and bitstream generation.
  \end{block}{}
  \scriptsize
  \verbatiminput{temp/synt}
\end{frame}

%\begin{frame}[fragile]{FPGA Synt (V) - vendors TCL files}
%  %\tiny
%  \begin{itemize}
%    \item The commands \textit{fpga\_device} and \textit{fpga\_file}, as well
%          as the value \textit{\$FPGA\_TOOL} are implemented in each vendor TCL
%          file.
%    \item If a project file exists, it is used and options.tcl is omitted.
%          The name of the project file must be different of the tool name
%          (it is used by the vendors TCL files).
%    \item Vendors TCL files generates log files called
%          TOOLNAME-OPERATION-OPTIMIZATION.log, where OPERATION could be syn or
%          imp and OPTIMIZATION could be none, area, power and speed.
%          Examples are \textit{ise-syn-area.log} and \textit{quartus2-imp-speed.log}.
%  \end{itemize}
%\end{frame}

%% FPGA Prog %%%%%%%%%%%%%%%%%%%%%%%%%%%%%%%%%%%%%%%%%%%%%%%%%%%%%%%%%%%%%%%%%%%%%%%%%%%%%%%%%%%%%%

\section{FPGA Prog}

\begin{frame}[fragile]{FPGA Prog (I)}
  \begin{block}{}
    If you have a bistream, FPGA Prog can be used to transfer it to a FPGA or memory.
  \end{block}{}
  \tiny
  \verbatiminput{temp/prog1}
\end{frame}

\begin{frame}[fragile]{FPGA Prog (II)}
  \tiny
  \verbatiminput{temp/prog2}
\end{frame}

%%%%%%%%%%%%%%%%%%%%%%%%%%%%%%%%%%%%%%%%%%%%%%%%%%%%%%%%%%%%%%%%%%%%%%%%%%%%%%%%%%%%%%%%%%%%%%%%%%%

\section{License}

\begin{frame}{Tutorial License}
  \begin{figure}[!t]
    \includegraphics[width=0.2\textwidth]{../images/cc-by-sa.png}
  \end{figure}
  \centering
  This work is licensed under a Creative Commons Attribution-ShareAlike 4.0 International License.
  \url{https://creativecommons.org/licenses/by-sa/4.0/}
\end{frame}

\end{document}
