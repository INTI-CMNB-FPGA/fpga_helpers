\documentclass{beamer}

\mode<presentation> {
  %\usetheme{default}
  %\usetheme{AnnArbor}
  %\usetheme{Antibes}
  %\usetheme{Bergen}
  %\usetheme{Berkeley}
  %\usetheme{Berlin}
  \usetheme{Boadilla}
  %\usetheme{CambridgeUS}
  %\usetheme{Copenhagen}
  %\usetheme{Darmstadt}
  %\usetheme{Dresden}
  %\usetheme{Frankfurt}
  %\usetheme{Goettingen}
  %\usetheme{Hannover}
  %\usetheme{Ilmenau}
  %\usetheme{JuanLesPins}
  %\usetheme{Luebeck}
  %\usetheme{Madrid}
  %\usetheme{Malmoe}
  %\usetheme{Marburg}
  %\usetheme{Montpellier}
  %\usetheme{PaloAlto}
  %\usetheme{Pittsburgh}
  %\usetheme{Rochester}
  %\usetheme{Singapore}
  %\usetheme{Szeged}
  %\usetheme{Warsaw}

  %\usecolortheme{albatross}
  %\usecolortheme{beaver}
  %\usecolortheme{beetle}
  %\usecolortheme{crane}
  %\usecolortheme{dolphin}
  %\usecolortheme{dove}
  %\usecolortheme{fly}
  %\usecolortheme{lily}
  %\usecolortheme{orchid}
  %\usecolortheme{rose}
  %\usecolortheme{seagull}
  %\usecolortheme{seahorse}
  \usecolortheme{whale}
  %\usecolortheme{wolverine}

  %\setbeamertemplate{footline} % To remove the footer line in all slides uncomment this line
  %\setbeamertemplate{footline}[page number]
  % To replace the footer line in all slides with a simple slide count uncomment this line
  \setbeamertemplate{navigation symbols}{}
  % To remove the navigation symbols from the bottom of all slides uncomment this line
}

\usepackage{graphicx} % Allows including images
\usepackage{booktabs} % Allows the use of \toprule, \midrule and \bottomrule in tables
\usepackage{times}
\usepackage[T1]{fontenc}
\usepackage{verbatim}

%%%%%%%%%%%%%%%%%%%%%%%%%%%%%%%%%%%%%%%%%%%%%%%%%%%%%%%%%%%%%%%%%%%%%%%%%%%%%%%%%%%%%%%%%%%%%%%%%%%
\newcommand{\python}         {\textit{Python}}
\newcommand{\bash}           {\textit{Bash}}
\newcommand{\debian}         {\textit{Debian GNU/Linux}}
\newcommand{\fpgasetup}      {\textit{fpga\_setup}}
\newcommand{\fpgawizard}     {\textit{fpga\_wizard}}
\newcommand{\fpgasynt}       {\textit{fpga\_synt}}
\newcommand{\fpgaprog}       {\textit{fpga\_prog}}
\newcommand{\fpgadeps}       {\textit{fpga\_deps}}
\newcommand{\console}        {\textit{console}}
\newcommand{\makefile}       {\textit{Makefile}}

\newcommand{\shellcmd}[1]{\\\indent\indent\texttt{\scriptsize #1}\\}
%%%%%%%%%%%%%%%%%%%%%%%%%%%%%%%%%%%%%%%%%%%%%%%%%%%%%%%%%%%%%%%%%%%%%%%%%%%%%%%%%%%%%%%%%%%%%%%%%%%

\title[FPGA Helpers]{Tutorial: FPGA Helpers v0.3.0}

\author{
  Rodrigo A. Melo\\
  \textit{rodrigomelo9@gmail.com}\\
  \textit{ar.linkedin.com/in/rodrigoalejandromelo}
}

%%%%%%%%%%%%%%%%%%%%%%%%%%%%%%%%%%%%%%%%%%%%%%%%%%%%%%%%%%%%%%%%%%%%%%%%%%%%%%%%%%%%%%%%%%%%%%%%%%%

\begin{document}

\begin{frame}
  \titlepage
\end{frame}

\section{Intro}

\begin{frame}{FPGA Helpers}
  \begin{exampleblock}{What is this?}
    A Free Software project which consist on a bunch of scripts which helps to use FPGA development
    tools in a vendor independent way.
  \end{exampleblock}
  \begin{block}{Development}
    \python\ and \bash\ scripts over \debian.
  \end{block}
\end{frame}

\begin{frame}{Current components}
  \begin{itemize}
    \item \textbf{\fpgasetup:} set the system to execute the vendor's tool.
    \item \textbf{\fpgawizard:} generates options.tcl and Makefile (project files).
    \item \textbf{\fpgasynt:} run synthesis based on the project file of the vendor's tool.
    \item \textbf{\fpgaprog:} transfer a bitstream to a FPGA or memory.
    \item \textbf{\fpgadeps:} collect HDL files of the project [WIP].
  \end{itemize}
\end{frame}

%% FPGA Setup %%%%%%%%%%%%%%%%%%%%%%%%%%%%%%%%%%%%%%%%%%%%%%%%%%%%%%%%%%%%%%%%%%%%%%%%%%%%%%%%%%%%%

\section{FPGA Setup}

\begin{frame}[fragile]{FPGA Setup (I)}
  \begin{block}{}
    The \makefile\ which run the Tcl scripts for synthesis and programming assumes that the
    vendor's tool is well configured and available in the system path.
    It could be done manually, automated in for example \textit{.bashrc} or using \fpgasetup.
  \end{block}{}
  \scriptsize
  \verbatiminput{temp/fpgasetup1.txt}
\end{frame}

\begin{frame}[fragile]{FPGA Setup (II)}
  It also can be run interactively:
  \scriptsize
  \verbatiminput{temp/fpgasetup2.txt}
  \normalsize
  About configurations:
  \begin{itemize}
    \item Are related to paths and licenses servers (when needed)
    \item Has default values and are filled interactively
    \item The \textit{tab key} can be used to autocomplete paths and files
    \item Data is saved on <HOME>/.fpga\_helpers file
  \end{itemize}
\end{frame}

%% FPGA Wizard %%%%%%%%%%%%%%%%%%%%%%%%%%%%%%%%%%%%%%%%%%%%%%%%%%%%%%%%%%%%%%%%%%%%%%%%%%%%%%%%%%%%

\section{FPGA Wizard}

\begin{frame}[fragile]{FPGA Wizard (I)}
  \begin{block}{}
    Generates options.tcl and Makefile (project files).
  \end{block}{}
  \tiny
  \begin{verbatim}
$ fpga_wizard.py 
fpga_wizard is a member of FPGA Helpers v0.3.0

Select TOOL to use [vivado]
EMPTY for default option. TAB for autocomplete. Your selection here > 

Where to get (if exists) or put Tcl files? [../tcl]
EMPTY for default option. TAB for autocomplete. Your selection here > project

Top Level file? [None]
EMPTY for default option. TAB for autocomplete. Your selection here > top_file.vhdl

Add files to the project (EMPTY to FINISH):
* Path to the file [FINISH]:
EMPTY for default option. TAB for autocomplete. Your selection here > core_file.vhdl
* In library [None]:
EMPTY for default option. TAB for autocomplete. Your selection here > LIB_NAME
* Path to the file [FINISH]:
EMPTY for default option. TAB for autocomplete. Your selection here > package_file.vhdl
* In library [None]:
EMPTY for default option. TAB for autocomplete. Your selection here > LIB_NAME
* Path to the file [FINISH]:
EMPTY for default option. TAB for autocomplete. Your selection here > 

Board to be used? [None]
EMPTY for default option. TAB for autocomplete. Your selection here > 
  \end{verbatim}
\end{frame}

\begin{frame}[fragile]{FPGA Wizard (II)}
  \tiny
  \begin{verbatim}
Specify the used FPGA [UNKNOWN]
EMPTY for default option. TAB for autocomplete. Your selection here > XC6SLX9-2-CSG324

Specify the FPGA position [1]
EMPTY for default option. TAB for autocomplete. Your selection here > 

Specify an attached SPI [None]
EMPTY for default option. TAB for autocomplete. Your selection here > 

Specify an attached BPI [None]
EMPTY for default option. TAB for autocomplete. Your selection here > 

fpga_wizard (INFO): directory project was created
fpga_wizard (INFO): Makefile was copy to project
fpga_wizard (INFO): synthesis.tcl was copy to project
fpga_wizard (INFO): programming.tcl was copy to project
fpga_wizard (INFO): Makefile and options.tcl were generated
  \end{verbatim}
  \begin{columns}
    \column{0.5\textwidth}
      options.tcl
      \begin{verbatim}
set fpga_name XC6SLX9-2-CSG324
set fpga_pos  1

fpga_device   $fpga_name

fpga_file     core_file.vhdl    -lib LIB_NAME
fpga_file     package_file.vhdl -lib LIB_NAME
fpga_file     top_file.vhdl     -top TOP_NAME
      \end{verbatim}
    \column{0.5\textwidth}
      Makefile
      \begin{verbatim}
#!/usr/bin/make
#Generated with fpga_wizard v0.3.0

TOOL    = vivado
TCLPATH = project
include $(TCLPATH)/Makefile
      \end{verbatim}
  \end{columns}
\end{frame}

%% FPGA Synt %%%%%%%%%%%%%%%%%%%%%%%%%%%%%%%%%%%%%%%%%%%%%%%%%%%%%%%%%%%%%%%%%%%%%%%%%%%%%%%%%%%%%%

\section{FPGA Synt}

\begin{frame}[fragile]{FPGA Synt (I)}
  \begin{block}{}
    You can make a project using the vendor's tool GUI, and after that, use FPGA Synt to run
    synthesis, implementation and bitstream generation.
  \end{block}{}
  \scriptsize
  \verbatiminput{temp/fpgasynt.txt}
\end{frame}

%\begin{frame}[fragile]{FPGA Synt (II) - Examples}
%  \small
%  \verbatiminput{temp/fpgasynt_ex1.txt}
%  \verbatiminput{temp/fpgasynt_ex2.txt}
%\end{frame}

%\begin{frame}[fragile]{FPGA Synt (III) - Makefile}
%  \tiny
%  \verbatiminput{temp/makefile.txt}
%  \begin{alertblock}{}
%    \small
%    The \makefile\ uses vendors tools, so you could run \fpgasetup\ when open
%    a new \console.
%  \end{alertblock}
%\end{frame}

%\begin{frame}[fragile]{FPGA Synt (V) - vendors TCL files}
%  %\tiny
%  \begin{itemize}
%    \item The commands \textit{fpga\_device} and \textit{fpga\_file}, as well
%          as the value \textit{\$FPGA\_TOOL} are implemented in each vendor TCL
%          file.
%    \item If a project file exists, it is used and options.tcl is omitted.
%          The name of the project file must be different of the tool name
%          (it is used by the vendors TCL files).
%    \item Vendors TCL files generates log files called
%          TOOLNAME-OPERATION-OPTIMIZATION.log, where OPERATION could be syn or
%          imp and OPTIMIZATION could be none, area, power and speed.
%          Examples are \textit{ise-syn-area.log} and \textit{quartus2-imp-speed.log}.
%  \end{itemize}
%\end{frame}

%% FPGA Prog %%%%%%%%%%%%%%%%%%%%%%%%%%%%%%%%%%%%%%%%%%%%%%%%%%%%%%%%%%%%%%%%%%%%%%%%%%%%%%%%%%%%%%

\section{FPGA Prog}

\begin{frame}[fragile]{FPGA Prog (I)}
  \begin{block}{}
    If you have a bistream, FPGA Prog can be used to transfer it to a FPGA or memory.
  \end{block}{}
  \scriptsize
  \begin{verbatim}
$ fpga_prog --help
usage: fpga_prog [-h] [-v] [-t TOOL] [-d DEVICE] [-b BOARDNAME] [-m MEMNAME]
                 [-p POSITION] [-w WIDTH]
                 [BITSTREAM]

Transfers a BitStream to a device.

positional arguments:
  BITSTREAM             bitstream to be transferred

optional arguments:
  -h, --help            show this help message and exit
  -v, --version         show program's version number and exit
  -t TOOL, --tool TOOL  name of the vendor tool to be used
                        (ise|quartus|libero|vivado) [vivado]
  \end{verbatim}
\end{frame}

\begin{frame}[fragile]{FPGA Prog (II)}
  \scriptsize
  \begin{verbatim}
  -d DEVICE, --device DEVICE
                        type of the target device
                        (fpga|spi|bpi|xcf|detect|unlock) [fpga]
  -b BOARDNAME, --board BOARDNAME
                        name of a supported board (note: if you use the board
                        option, -p, -m and -w will be overwritten) [None]
  -m MEMNAME, --memname MEMNAME
                        name of the memory target device [UNDEFINED]
  -p POSITION, --position POSITION
                        positive number which represents the POSITION of the
                        device in the JTAG chain [1]
  -w WIDTH, --width WIDTH
                        positive number which representes the WIDTH of bits of
                        the target memory (1, 2, 4, 8, 16, 32, 64) [1]

Supported boards: xilinx_ml605, microsemi_m2s090ts, gaisler_xc6s,
xilinx_sp601, avnet_s6micro, digilent_atlys, terasic_de0nano
  \end{verbatim}
\end{frame}

%%%%%%%%%%%%%%%%%%%%%%%%%%%%%%%%%%%%%%%%%%%%%%%%%%%%%%%%%%%%%%%%%%%%%%%%%%%%%%%%%%%%%%%%%%%%%%%%%%%

\section{License}

\begin{frame}{Tutorial License}
  \begin{figure}[!t]
    \includegraphics[width=0.2\textwidth]{../images/cc-by-sa.png}
  \end{figure}
  \centering
  This work is licensed under a Creative Commons Attribution-ShareAlike 4.0 International License.
  \url{https://creativecommons.org/licenses/by-sa/4.0/}
\end{frame}

\end{document}
