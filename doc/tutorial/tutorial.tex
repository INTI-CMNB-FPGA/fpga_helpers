\documentclass{beamer}

\mode<presentation> {
  %\usetheme{default}
  %\usetheme{AnnArbor}
  %\usetheme{Antibes}
  %\usetheme{Bergen}
  %\usetheme{Berkeley}
  %\usetheme{Berlin}
  \usetheme{Boadilla}
  %\usetheme{CambridgeUS}
  %\usetheme{Copenhagen}
  %\usetheme{Darmstadt}
  %\usetheme{Dresden}
  %\usetheme{Frankfurt}
  %\usetheme{Goettingen}
  %\usetheme{Hannover}
  %\usetheme{Ilmenau}
  %\usetheme{JuanLesPins}
  %\usetheme{Luebeck}
  %\usetheme{Madrid}
  %\usetheme{Malmoe}
  %\usetheme{Marburg}
  %\usetheme{Montpellier}
  %\usetheme{PaloAlto}
  %\usetheme{Pittsburgh}
  %\usetheme{Rochester}
  %\usetheme{Singapore}
  %\usetheme{Szeged}
  %\usetheme{Warsaw}

  %\usecolortheme{albatross}
  %\usecolortheme{beaver}
  %\usecolortheme{beetle}
  %\usecolortheme{crane}
  %\usecolortheme{dolphin}
  %\usecolortheme{dove}
  %\usecolortheme{fly}
  %\usecolortheme{lily}
  %\usecolortheme{orchid}
  %\usecolortheme{rose}
  %\usecolortheme{seagull}
  %\usecolortheme{seahorse}
  \usecolortheme{whale}
  %\usecolortheme{wolverine}

  %\setbeamertemplate{footline} % To remove the footer line in all slides uncomment this line
  %\setbeamertemplate{footline}[page number]
  % To replace the footer line in all slides with a simple slide count uncomment this line
  \setbeamertemplate{navigation symbols}{}
  % To remove the navigation symbols from the bottom of all slides uncomment this line
}

\usepackage{graphicx} % Allows including images
\usepackage{booktabs} % Allows the use of \toprule, \midrule and \bottomrule in tables
\usepackage{times}
\usepackage[T1]{fontenc}

%%%%%%%%%%%%%%%%%%%%%%%%%%%%%%%%%%%%%%%%%%%%%%%%%%%%%%%%%%%%%%%%%%%%%%%%%%%%%%%%%%%%%%%%%%%%%%%%%%%
\newcommand{\python}         {\textit{Python}}
\newcommand{\bash}           {\textit{Bash}}
\newcommand{\debian}         {\textit{Debian GNU/Linux}}
\newcommand{\fpgasetup}      {\textit{fpga\_setup}}
\newcommand{\fpgasynt}       {\textit{fpga\_synt}}
\newcommand{\fpgaprog}       {\textit{fpga\_prog}}
\newcommand{\console}        {\textit{console}}
\newcommand{\makefile}       {\textit{Makefile}}

\newcommand{\shellcmd}[1]{\\\indent\indent\texttt{\scriptsize #1}\\}
%%%%%%%%%%%%%%%%%%%%%%%%%%%%%%%%%%%%%%%%%%%%%%%%%%%%%%%%%%%%%%%%%%%%%%%%%%%%%%%%%%%%%%%%%%%%%%%%%%%

\title[FPGA Helpers]{Tutorial: FPGA Helpers v0.1.1}

\author{
  Rodrigo A. Melo\\
  \textit{rodrigomelo9@gmail.com}\\
  \textit{ar.linkedin.com/in/rodrigoalejandromelo}
}

%%%%%%%%%%%%%%%%%%%%%%%%%%%%%%%%%%%%%%%%%%%%%%%%%%%%%%%%%%%%%%%%%%%%%%%%%%%%%%%%%%%%%%%%%%%%%%%%%%%

\begin{document}

\begin{frame}
  \titlepage
\end{frame}

\section{Intro}

\begin{frame}{FPGA Helpers}
  \begin{exampleblock}{What is this?}
    A bunch of scripts to use FPGA development tools in a vendor independent way.
  \end{exampleblock}
  \begin{block}{Development}
    \python\ and \bash\ scripts over \debian.
  \end{block}
\end{frame}

\begin{frame}{Current components}
  \begin{itemize}
    \item \textbf{\fpgasetup:} used to set the system to execute the tools of the vendors.
    \item \textbf{\fpgasynt:} generates a Makefile and two TCL files, which are used to
      configure and execute a synthesis project.
    \item \textbf{\fpgaprog:} used to transfer a bitstream to a FPGA or memory.
  \end{itemize}
\end{frame}

%% FPGA Setup %%%%%%%%%%%%%%%%%%%%%%%%%%%%%%%%%%%%%%%%%%%%%%%%%%%%%%%%%%%%%%%%%%%%%%%%%%%%%%%%%%%%%

\begin{frame}[fragile]{FPGA Setup (I)}
  \begin{columns}
    \column{0.75\textwidth}
      \begin{block}{}
        \small
        \fpgaprog\ and the \makefile generated by \fpgasynt\ uses vendors tools.
        They assumes that are well installed and configured (license) and the
        binaries must be in the system path. It could be done manually,
        automated in for example \textit{.bashrc} or using \fpgasetup.
      \end{block}{}
    \column{0.25\textwidth}
      \small Supported tools:
      \begin{itemize}
        \tiny
        \item Xilinx ISE
        \item Xilinx Vivado
        \item Altera Quartus2
        \item Microsemi Libero-SoC
      \end{itemize}
  \end{columns}
  \scriptsize
  \begin{verbatim}
$ fpga_setup --help
FPGA Setup Help:
* Execute fpga_setup without arguments to run interactively.
* Available options when run with arguments (choose only one):
--config     : configurations related with vendors tools
--all        : set to use all the available vendors tools
--ise        : set to use ISE (Xilinx)
--vivado     : set to use Vivado (Xilinx)
--quartus2   : set to use Quartus2 (Altera)
--libero-soc : set to use Libero-SoC (Microsemi)
  \end{verbatim}
\end{frame}

\begin{frame}[fragile]{FPGA Setup (II)}
  It can run interactively:
  \scriptsize
  \begin{verbatim}
$ fpga_setup
c. configurations related with vendors tools
0. set to use all the available vendors tools
1. set to use ISE (Xilinx)
2. set to use Vivado (Xilinx)
3. set to use Quartus2 (Altera)
4. set to use Libero-SoC (Microsemi)
  \end{verbatim}
  \normalsize
  About configurations:
  \begin{itemize}
    \item Are related to paths and licenses servers (when needed)
    \item Has default values and are filled interactively
    \item The \textit{tab key} can be used to autocomplete paths and files
    \item Data is saved on <HOME>/.fpga\_helpers file
  \end{itemize}
\end{frame}

%% FPGA Synt %%%%%%%%%%%%%%%%%%%%%%%%%%%%%%%%%%%%%%%%%%%%%%%%%%%%%%%%%%%%%%%%%%%%%%%%%%%%%%%%%%%%%%

\begin{frame}[fragile]{FPGA Synt (I)}
  \begin{columns}
    \column{0.75\textwidth}
      \begin{block}{}
        \small
        It generates a \makefile\ a TCL file for options and one or more TCL files
        with vendor specific commands to run the synthesis process.
      \end{block}{}
    \column{0.25\textwidth}
      \small Supported tools:
      \begin{itemize}
        \tiny
        \item Xilinx ISE
        \item Xilinx Vivado
        \item Altera Quartus2
      \end{itemize}
  \end{columns}
  \scriptsize
  \begin{verbatim}
$ fpga_synt --help
usage: fpga_synt.py [-h] [-v] [-b BOARDNAME|BOARDFILE] [TOOLNAME]

Generates files to make a Synthesis.

positional arguments:
  TOOLNAME              Name of the vendor tool to be used
                        [ise|vivado|quartus2|all]

optional arguments:
  -h, --help            show this help message and exit
  -v, --version         show program's version number and exit
  -b BOARDNAME|BOARDFILE, --board BOARDNAME|BOARDFILE
                        Name of a supported board or file (.yaml) of a
                        new/custom board

Supported boards: avnet_s6micro, digilent_atlys, gaisler_xc6s,
terasic_de0nano, xilinx_sp601
  \end{verbatim}
\end{frame}

%% FPGA Prog %%%%%%%%%%%%%%%%%%%%%%%%%%%%%%%%%%%%%%%%%%%%%%%%%%%%%%%%%%%%%%%%%%%%%%%%%%%%%%%%%%%%%%

%%%%%%%%%%%%%%%%%%%%%%%%%%%%%%%%%%%%%%%%%%%%%%%%%%%%%%%%%%%%%%%%%%%%%%%%%%%%%%%%%%%%%%%%%%%%%%%%%%%

\begin{frame}{Tutorial License}
  \begin{figure}[!t]
    \includegraphics[width=0.2\textwidth]{../images/cc-by-sa.png}
  \end{figure}
  \centering
  This work is licensed under a Creative Commons Attribution-ShareAlike 4.0 International License.
  \url{https://creativecommons.org/licenses/by-sa/4.0/}
\end{frame}

\end{document}
